\documentclass[]{scrartcl}
\usepackage[utf8]{inputenc}
\usepackage[spanish]{babel}
\usepackage{graphicx}
\usepackage{color}
\usepackage[usenames,dvipsnames,svgnames,table]{xcolor}
\usepackage{hyperref}
\usepackage{url}
\hypersetup{
	colorlinks   = true,
	citecolor    = gray,
	urlcolor     = darkgray,
	linkcolor	 = darkgray
}

%opening
\title{Sistemas Inteligentes para la Gestión en la Empresa\\-\\Práctica 2: Deep Learning para multi-clasificación}
\author{Carlos Cobos Suárez\\Adrián Morente Gabaldón}

\begin{document}

\maketitle
\newpage
\tableofcontents
\newpage

\section{Introducción}

En este proyecto vamos a realizar diversas aproximaciones al tratamiento de técnicas de \textbf{aprendizaje automático} con el lenguaje \textbf{\textit{R}}. Tras ponernos en situación con los fundamentos teóricos necesarios para entender el desarrollo realizado, comentaremos las soluciones pensadas y las que finalmente se hayan implementado; realizando después una discusión de los resultados obtenidos.\\

El conjunto de datos (o \textit{dataset}) que utilizaremos se corresponde con el de \textbf{\textit{PetFinder.my Adoption Prediction}}, y podemos encontrarlo en la plataforma \href{https://kaggle.com}{Kaggle} \cite{petfinder-dataset}. Los datos contenidos clasifican un histórico de perros y gatos alojados en centros de adopción de animales, con diversas características y almacenando el \textbf{tiempo de adopción} de cada una de estas mascotas.\\

El trabajo a desarrollar en este proyecto versa sobre el \textbf{entrenamiento de modelos de predicción} que, a partir de imágenes o de un conjunto de características de animales, clasifiquen éstos por tiempo de adopción. Utilizaremos \textbf{cinco categorías} (las cuales se ordenan según el tiempo de adopción en orden ascendente).\\

Dado que para el entrenamiento se dispone de un gran conjunto de datos ya clasificado, realizaremos algún \textbf{histograma}, lo que trata de un gráfico de barras que muestra el cardinal del conjunto de datos de cada una de las distintas categorías mencionadas.

\section{Fundamentos teóricos}

	\subsection{\textit{Deep Learning}}
	
		\subsubsection{Modelos de clasificación}
		
		\subsubsection{Redes neuronales}
		
		\subsubsection{Redes convolucionales}
		
		\subsubsection{Técnicas de binarización}
		
		\subsubsection{\textit{Ensembles}}

\section{Descripción de las redes empleadas}

	\subsection{Primera aproximación - Red vista en clase}
	
	\subsection{Segunda aproximación - Balanceo de clases}
	
	\subsection{Tercera aproximación - \textit{Data augmentation}}
	
\section{Discusión de resultados}

\section{Conclusiones}

\bibliographystyle{plain}
\bibliography{sources}

\end{document}
